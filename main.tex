\documentclass[12pt]{article}
\usepackage{times}
\usepackage{geometry}
\geometry{letterpaper, portrait, margin=1in}
\usepackage[utf8]{inputenc}
\usepackage{enumitem,amssymb}
\usepackage{ragged2e}
\usepackage{graphicx}
\usepackage[usenames]{xcolor} %used for font color              
\definecolor{xlinkcolor}{cmyk}{1,1,0,0}
\usepackage{url}
\usepackage[
 colorlinks=true,    % false: boxed links; true: colored links 
 linkcolor=xlinkcolor,     % color of internal links            
 citecolor=xlinkcolor,     % color of links to bibliography
 filecolor=xlinkcolor,  % color of file links 
 urlcolor=xlinkcolor,      % color of external link
 final=true
]{hyperref}
\newlist{thematic}{itemize}{8}
\setlist[thematic]{label=$\square$}
\usepackage{pifont}
\newcommand{\cmark}{\ding{51}}%
\newcommand{\xmark}{\ding{55}}%
\newcommand{\done}{\rlap{$\square$}{\raisebox{2pt}{\large\hspace{1pt}\cmark}}%
\hspace{-2.5pt}}
\newcommand{\wontfix}{\rlap{$\square$}{\large\hspace{1pt}\xmark}}

\begin{document}
\raggedright
\huge
Astro2020 Science White Paper \linebreak

LSST: the Dark Matter Telescope\linebreak
\normalsize

\noindent \textbf{Thematic Areas:} \hspace*{60pt} $\square$ Planetary Systems \hspace*{10pt} $\square$ Star and Planet Formation \hspace*{20pt}\linebreak
$\blacksquare$ Formation and Evolution of Compact Objects \hspace*{31pt} $\blacksquare$ Cosmology and Fundamental Physics \linebreak
  $\square$  Stars and Stellar Evolution \hspace*{1pt} $\square$ Resolved Stellar Populations and their Environments \hspace*{40pt} \linebreak
  $\square$    Galaxy Evolution   \hspace*{45pt} $\square$             Multi-Messenger Astronomy and Astrophysics \hspace*{65pt} \linebreak
  
\textbf{Principal Author:}

Name:	
 \linebreak						
Institution:  
 \linebreak
Email: 
 \linebreak
Phone:  
 \linebreak
 
\textbf{Co-authors:} (names and institutions)
  \linebreak

\textbf{Abstract  (optional):}
Astrophysical and cosmological observations currently provide the only robust, empirical measurements of dark matter. Future observations with Large Synoptic Survey Telescope (LSST) will provide necessary guidance for the experimental dark matter program. This white paper represents a community effort to summarize the science case for studying the fundamental physics of dark matter with LSST. We discuss how LSST will inform our understanding of the fundamental properties of dark matter, such as particle mass, self-interaction strength, non-gravitational couplings to the Standard Model, and compact object abundances. Additionally, we discuss the ways that LSST will complement other experiments to strengthen our understanding of the fundamental characteristics of dark matter. More information on the LSST dark matter effort can be found at \href{https://lsstdarkmatter.github.io/}{https://lsstdarkmatter.github.io/}.

\pagebreak

\section{Executive Summary}

\section{Dark Matter Models}

\section{Dark Matter Probes}

\section{Complementarity}

\section{Discovery Potential}



Insert your white paper text here (max of five pages including figures).

\pagebreak
\section{References}



\end{document}

